In recent years, the market share of smartphones has increased rapidly. We have been surrounded by the computational power as well as a variety of sensors of mobile devices. As a result, a number of novel applications and location-based services have been developed. People have gradually found the limitations of the global positioning system (GPS) and developed all kinds of different indoor positioning systems on top of Wi-Fi~\cite{ferris2007wifi}, Bluetooth~\cite{bekkelien2012bluetooth}, light digital pulse~\cite{ganick2013light}, and magnetic fields inside a building~\cite{racoma2013indooratlas}. In addition to these approaches, some visual-based positioning systems~\cite{mulloni2009indoor}~\cite{kawaji2010image} have also been proposed.

In this research, we implemented a visual-based positioning system on Android-based smartphone. Our system makes use of the built-in camera and sensors to compute the relative distance and angle between the reference point and the camera. The system runs on a standalone Android smartphone with no assisted devices. It has a very low cost and is very easy to deploy to consumers who have a mobile device. The experiment results show that our proposed visual-based positioning system is both accurate and stable for applications such as indoor positioning navigation, virtual augmentations, and location-based services.
